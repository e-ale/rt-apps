\section{Evaluating Real-Time Systems}

\begin{frame}
   {How to Evaluate a Real-Time System?}

   \begin{itemize}
      \item
      Use the \textbf{cyclictest} tool. (Part of the \textbf{rt-tests} package.)
      \begin{itemize}
         \item
         measures/tracks latencies from hardware interrupt to userspace
         \item
         run at the priority level to evaluate
      \end{itemize}
      \item
      Generate worst case system loads.
      \begin{itemize}
         \item
         scheduling load: the \textbf{hackbench} tool
         \item
         interrupt load: flood pinging with ''\texttt{ping -f}''
         \item
         serial/network load: ''\texttt{top -d 0}'' via console and network shells
         \item
         memory loads: OOM killer invocations
         \item
         various load scenarios: the \textbf{stress-ng} tool
      \end{itemize}
   \end{itemize}

\end{frame}

\cprotect\note{

   Loads should be generated for hardware devices that will be in use on
   the device. If it has USB, generate USB loads. If it has a display,
   generate graphic loads. Etc...

   It is important that \textbf{cyclictest} is run at the correct
   priority for the tests. Otherwise the results may be worse than they
   should be. If testing the overall real-time performance for the system,
   it is common to start \textbf{cyclictest} with:
   \begin{raw}
cyclictest -S -m -p 98 --secaligned
   \end{raw}

   Note that real-time priority 99 should never be used by userspace
   software. This highest priority is used by critical kernel tasks that
   should never be interrupted by userspace software.

   If you are unhappy with the latency results (and your are sure that
   you performed the test correctly), be aware that there are many
   kernel options available to tune the real-time performance of Linux.

}
