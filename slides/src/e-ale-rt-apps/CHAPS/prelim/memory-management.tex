\section{Understanding Memory Management}

\begin{frame}
   {Page Faulting}

   By default, physical memory pages are mapped to the virtual
   address space \textbf{on demand}. This allows features such as
   over-commitment and it affects \textbf{all} virtual memory of
   a process:

   \begin{itemize}
      \item
      text segment
      \item
      initialized data segment
      \item
      uninitialized data segment
      \item
      stack(s)
      \item
      heap
   \end{itemize}

\end{frame}

\cprotect\note{

   As an example, when a task allocates memory, the kernel will
   tell the task that the memory has been allocated and return a
   virtual address pointer to such memory. However, no actual
   physical memory has been reserved. It is not until the task
   starts reading/writing to the memory, that the kernel will
   actually find physical memory to map to it.

   When virtual memory is accessed that has no physical memory
   mapping, this generates a page fault in the memory management
   unit (MMU). The kernel's exception handler will catch this
   and scramble to find a free page of physical memory to map.
   If no physical memory is available, and there are no mapped
   pages that can be paged out, the kernel will invoke the out
   of memory (OOM) killer.

   In summary, malloc's are cheap and first accesses are
   expensive.

}
