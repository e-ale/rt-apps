\section{Limiting/Isolating CPUs}

\begin{frame}
   {CPU Affinity}

   \begin{itemize}
      \item
      Each task has its own CPU affinity mask, specifying which CPUs
      it may be scheduled on.
      \item
      Boot parameters are available to set default masks for all tasks
      (including the kernel's own tasks).
      \item
      A CPU affinity mask for routing individual hardware interrupt
      handling is also available.
   \end{itemize}

\end{frame}

\cprotect\note{

   By isolating real-time tasks on CPUs, the real-time performance can
   be further increased. Less interruptions usually translates to less
   latency. Just be aware that SMP systems usually have multiple CPUs
   sharing caches. Unfortunately this means that non-real-time tasks
   can influence a real-time task, even if that task is completely
   isolated on its own CPU:

   The CPU affinity mask can be set using the \textbf{taskset} command:
   \begin{raw}
taskset [options] mask command [arg]...
taskset [options] -p [mask] pid
   \end{raw}

   ... or in code:
   \begin{raw}
#define _GNU_SOURCE
#include <sched.h>

cpu_set_t set;

CPU_ZERO(&set);
CPU_SET(0, &set);
CPU_SET(1, &set);
sched_setaffinity(pid, CPU_SETSIZE, &set);
   \end{raw}

   Kernel parameters related to CPU isolation:

   \begin{itemize}
      \item
      maxcpus=\textit{n}: limits the kernel to bring up \textit{n} CPUs
      \item
      isolcpus=\textit{cpulist}: specify CPUs to isolate from disturbances
   \end{itemize}

   The default CPU affinity for routing hardware interrupts when new
   interrupt handlers are registered can be viewed/set in:

   \texttt{/proc/irq/default\_smp\_affinity}.

   The CPU affinity for routing hardware interrupts for already registered
   interrupt handlers can be viewed/set in:

   \texttt{/proc/irq/}\textit{irq-number}\texttt{/smp\_affinity}.

}

