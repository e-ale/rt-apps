\section{Locking}

\begin{frame}
   {Synchronization}

   \begin{itemize}
      \item
      \textbf{Use the pthread\_mutex as the lock}. These objects
      have owners (unlike semaphores) so the kernel can more
      intelligently choose which processes to schedule.
      \item
      \textbf{Activate priority inheritance}. Unfortunately this
      is not the default.
      \item
      Activate shared and robustness features \textbf{if} the lock
      is accessed by multiple processes in shared memory.
   \end{itemize}

\end{frame}

\cprotect\note{

   Here is a trivial example showing the syntax and semantics
   of lock initialization and usage.
   \begin{raw}
#include <pthread.h>

pthread_mutex_t lock;
pthread_mutexattr_t mattr;

pthread_mutexattr_init(&mattr);
pthread_mutexattr_setprotocol(&mattr, PTHREAD_PRIO_INHERIT);
pthread_mutex_init(&lock, &mattr);

pthread_mutex_lock(&lock);
/* do critical work */
pthread_mutex_unlock(&lock);

pthread_mutex_destroy(&lock);
   \end{raw}

   The feature \textbf{PTHREAD\_PROCESS\_SHARED} should only be set
   if the lock resides in shared memory. Locks marked with this
   feature have additional overhead because they cannot use the
   fast userspace mutex (futex) implementation.

   Be aware that activating \textbf{PTHREAD\_MUTEX\_ROBUST} introduces
   new semantics for \textbf{phread\_mutex\_lock()}. Make sure you fully
   understand its usage, otherwise you may have broken synchronization.

}
