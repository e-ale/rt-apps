\section{Real-Time API}

\begin{frame}
   {POSIX}

   \begin{itemize}
      \item
      Linux real-time features are implemented using the POSIX
      standard API. Most developers are already comfortable with
      this interface.
      \item
      No exotic libraries.
      \item
      No exotic objects.
      \item
      No exotic functions.
      \item
      No exotic semantics.
   \end{itemize}

\end{frame}

\cprotect\note{

   Because the POSIX API is used, application developers can usually
   learn to write real-time applications very quickly. Also, since
   most Linux software is already written for POSIX, there is often
   little effort in combining existing code bases with real-time
   development.

   The real-time API for Linux is essentially contained in the
   \texttt{sched.h}, \texttt{time.h}, and \texttt{pthread.h} header
   files.

}
