\section{Controlling Memory}

\begin{frame}
   {Avoid Page Faults}

   \begin{itemize}
      \item
      \textbf{Tune glibc's malloc} to avoid memory mapping as a
      form of memory allocation. mmap'd memory cannot be reused
      after being freed.
      \item
      \textbf{Lock down allocated pages} so that they cannot be returned
      to the kernel. Hold on to what you've been given.
      \item
      \textbf{Prefault} the heap and the stack(s).
   \end{itemize}

\end{frame}

\cprotect\note{

   The \textbf{malloc} function of glibc can be tuned using the
   \textbf{mallopt} function. Below are the calls to disable memory
   mapping as a form of memory allocation. This is important
   because glibc cannot reuse freed memory if it has been allocated
   using \textbf{mmap}. Such memory is managed directly by the
   kernel.
   \begin{raw}
#include <malloc.h>

mallopt(M_TRIM_THRESHOLD, -1);
mallopt(M_MMAP_MAX, 0);
   \end{raw}

   Memory pages can be locked down with the \textbf{mlockall} function:
   \begin{raw}
#include <sys/mman.h>

mlockall(MCL_CURRENT | MCL_FUTURE);
   \end{raw}

   An example of heap prefaulting:
   \begin{raw}
#include <stdlib.h>
#include <unistd.h>

void prefault_heap(int size)
{
    char *dummy;
    int i;

    dummy = malloc(size);
    if (!dummy)
        return;

    for (i = 0; i < size; i += sysconf(_SC_PAGESIZE))
        dummy[i] = i;

    free(dummy);
}
   \end{raw}

   An example of stack prefaulting:
   \begin{raw}
#include <unistd.h>

#define MAX_SAFE_STACK (512 * 1024)

void prefault_stack(void)
{
    unsigned char dummy[MAX_SAFE_STACK];
    int i;

    for (i = 0; i < MAX_SAFE_STACK; i += sysconf(_SC_PAGESIZE))
        dummy[i] = i;
}
   \end{raw}

}
