\section{Signalling}

\begin{frame}
   {Conditional Variables}

   \begin{itemize}
      \item
      \textbf{Use pthread\_cond objects} for notifying tasks.
      These can be associated with pthread\_mutex objects to provide
      synchronized notification.
      \item
      Do not use signals (such as POSIX timers or the kill() function).
      They involve unclear and limited contexts, do not provide any
      synchronization, and are difficult to program correctly.
      \item
      Activate the shared feature \textbf{if} the conditional variable
      is accessed by multiple processes in shared memory.
      \item
      The sender should notify the receiver \textbf{before} releasing
      the lock associated with the conditional variable.
   \end{itemize}

\end{frame}

\cprotect\note{

   Here is an example showing the syntax and semantics
   of conditional variable initialization.
   \begin{raw}
#include <pthread.h>

pthread_condattr_t cattr;
pthread_cond_t cond;

pthread_condattr_init(&cattr);
pthread_cond_init(&cond, &cattr);
   \end{raw}

}

\begin{frame}
   {Signalling (code snippet)}

   \begin{raw}
#include <pthread.h>

pthread_mutex_t lock;
pthread_cond_t cond;
   \end{raw}

   Code of receiver:
   \begin{raw}
pthread_mutex_lock(&lock);
pthread_cond_wait(&cond, &lock);
/* we have been signaled */
pthread_mutex_unlock(&lock);
   \end{raw}

   Code of sender:
   \begin{raw}
pthread_mutex_lock(&lock);
/* do the work */
pthread_cond_broadcast(&cond);
pthread_mutex_unlock(&lock);
   \end{raw}

\end{frame}

\cprotect\note{

   There have been many discussions on the usefulness of sending the signal
   while holding the lock. Indeed, with the glibc and kernel implementations
   for SMP systems, this procedure does not provide any real advantages.

   However, using the POSIX API in this way allows for some (future)
   implementation where it can make an important difference.

   The main concern is what happens when the sender releases the lock. If
   a high priority task has been waiting, it is only fair that it gets
   the lock before a lower priority task. By signalling waiters before
   the lock has been released, the operating system is given the
   opportunity to prepare for who will get the lock \textbf{before} the
   lock has been released. This can provide the fairness necessary to
   avoid priority inversion.

}
