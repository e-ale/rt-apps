\section{Using Clocks}

\begin{frame}
   {The Monotonic Clock}

   \begin{itemize}
      \item
      Use the POSIX functions that allow clock specification. These
      begin with \textbf{clock\_}.
      \item
      \textbf{Choose CLOCK\_MONOTONIC}. This is a clock that cannot be
      set and represents monotonic time since some unspecified starting
      point.
      \item
      \textbf{Do not use CLOCK\_REALTIME}. This is a clock that represents
      the "real" time. For example, Tuesday 23 October 2018 17:00:00.
      This clock can be set by NTP, the user, etc.
      \item
      \textbf{Use absolute time values}. Calculating relative times is error prone
      because the calculation itself takes time.
   \end{itemize}

\end{frame}

\cprotect\note{

   When using \textbf{CLOCK\_MONOTONIC} and absolute times, a cyclical
   task becomes trivial to implement.
   \begin{raw}
#include <time.h>

#define CYCLE_TIME_NS (100 * 1000 * 1000) 
#define NSEC_PER_SEC (1000 * 1000 * 1000)

static void norm_ts(struct timespec *tv)
{
    while (tv->tv_nsec > NSEC_PER_SEC) {
        tv->tv_sec++;
        tv->tv_nsec -= NSEC_PER_SEC;
    }
}

void cycle_task(void)
{
    struct timespec tv;

    clock_gettime(CLOCK_MONOTONIC, &tv);

    while (1) {
        /* do the work */

        /* wait for next cycle */
        tv.tv_nsec += CYCLE_TIME_NS;
        norm_ts(&tv);
        clock_nanosleep(CLOCK_MONOTONIC, TIMER_ABSTIME, &tv, NULL);
    };
}
   \end{raw}

}
