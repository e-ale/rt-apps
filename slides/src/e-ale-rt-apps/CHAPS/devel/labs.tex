\clearpage\section{Labs}\begin{Lab}

\begin{exe} {Examine Page Fault Effects}

   The \textbf{pgflt} program demonstrates the effects of
   page faulting by performing several actions and tracking
   the time duration and number of page faults generated by
   those actions.

   The \textbf{pgflt} program has implemented the various memory
   controlling techniques presented here. Each of these techniques
   can be individually enabled and disabled to explore their effects.

   The test actions performed by \textbf{pgflt} are:

   \begin{itemize}
      \item
      allocate, set, and free 10MiB of memory
      \item
      call a function recursively to occupy a 7MiB stack
   \end{itemize}

   Each of the test actions are performed 4 times.

   \begin{raw}
usage: ./pgflt [opts-bitmask]
  opts-bits:
  0x01 = mallopt
  0x02 = mlockall
  0x04 = prefault-stack
  0x08 = prefault-heap
  0x10 = run tests

  0x10 = no rt tweaks + tests
  0x1f = full rt tweaks + tests
   \end{raw}

  Examples:
   \begin{raw}
./pgflt 0x10
./pgflt 0x1f
   \end{raw}

\end{exe}

\clearpage

\begin{exe} {Run/Investigate LED Master Program}

   The \textbf{ledmaster} program runs a real-time cyclic task that
   is updating a (different) LED every 50ms. Run it and observe its
   performance. Things to consider:

   \begin{itemize}
      \item
      the real-time priority (or non-real-time nice value) of the task
      \item
      the load on the system
      \item
      the performance of \textbf{cyclictest} at different real-time
      priorities
   \end{itemize}

   Priority tools: \texttt{chrt}, \texttt{nice}, \texttt{renice}

   Starting \textbf{ledmaster} with real-time priority:
   \begin{raw}
chrt -f 80 ./ledmaster
   \end{raw}

   Modifying the real-time priority of a running \textbf{ledmaster}:
   \begin{raw}
chrt -f -p 80 $(pidof ledmaster)
   \end{raw}

   Starting \textbf{ledmaster} with a non-real-time \textbf{nice} value:
   \begin{raw}
nice -n 19 ./ledmaster
   \end{raw}

\end{exe}

\clearpage

\begin{exe} {Run/Investigate LED Mirror Program}

   The \textbf{ledmaster} does not just set an LED with each
   cycle, but also stores the LED number and value of the
   most recently set LED into shared memory. In shared memory
   there is also a mutex and conditional variable, that is
   used to synchronize the data and signal any ''listeners''.
   After setting a value, \textbf{ledmaster} signals.

   The \textbf{ledmirror} program also sets an LED. However,
   rather than running cyclically, all it has available is the
   shared memory provided by the \textbf{ledmaster}.

   Run it and observe its performance. Things to consider:

   \begin{itemize}
      \item
      the real-time priority (or non-real-time nice value) of the task
      \item
      the load on the system
      \item
      its effect on the \textbf{ledmaster} task
   \end{itemize}

   Priority tools: \texttt{chrt}, \texttt{nice}, \texttt{renice}

   Starting \textbf{ledmirror} with real-time priority:
   \begin{raw}
chrt -f 70 ./ledmirror
   \end{raw}

   Modifying the real-time priority of a running \textbf{ledmirror}:
   \begin{raw}
chrt -f -p 70 $(pidof ledmirror)
   \end{raw}

   Starting \textbf{ledmirror} with a non-real-time \textbf{nice} value:
   \begin{raw}
nice -n 19 ./ledmirror
   \end{raw}

\end{exe}

\clearpage

\begin{exe} {Run the LED Priority Script}

   The \textbf{ledprio} script makes use of the thumbwheel
   driver implemented in the ''IIO and Input Drivers'' talk.
   It monitors for input events from the thumbwheel. When
   these events occur, the thumbwheel value is read and the
   priority of the \textbf{ledmirror} program is adjusted.
   The range is from \texttt{SCHED\_OTHER/nice=19} until
   \texttt{SCHED\_OTHER/nice=-20} and then finally
   \texttt{SCHED\_FIFO/rtprio=1}.

   Run all the components and play with the thumbwheel:
   \begin{raw}
chrt -f 80 ./ledmaster &
./ledmirror &
./ledprio
   \end{raw}

\end{exe}

\end{Lab}
