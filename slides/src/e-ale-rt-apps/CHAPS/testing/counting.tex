\section{Performance Counters and Events}

\begin{frame}
   {perf}

   \begin{itemize}
      \item
      \textbf{perf} is a tool that can count various types of hardware
      and software events.
      \item
      Some examples: CPU cycles, page faults, cache misses,
      context switches, scheduling events, ...
      \item
      It can help to identify performance issues with real-time tasks
      by showing if certain types of latency causing events are occurring.
   \end{itemize}

\end{frame}

\cprotect\note{

   Versions of \textbf{perf} may depend on certain kernel versions/features.
   For this reason it is important that the correct \textbf{perf} version
   is used. \textbf{perf} is part of the Linux kernel source, in
   \texttt{tools/perf}. It is best to use the version that comes from the
   same source as the running kernel.

   \textbf{perf} has many features and the help is spread across many
   man pages. However, \textbf{perf} can automatically open the correct man
   pages:
   \begin{raw}
perf help
perf help <command>
   \end{raw}

   List all supported events:
   \begin{raw}
perf list
   \end{raw}

   Count the number of CPU cycles and page faults on a system over 5 seconds:
   \begin{raw}
perf stat -a -e cpu-cycles -e page-faults sleep 5
   \end{raw}

   Count the number of context switches for the LED tools over 5 seconds:
   \begin{raw}
perf stat -a -e cpu-cycles -e context-switches -p $(pidof ledmaster),$(pidof ledmirror) sleep 5
   \end{raw}

   View the top symbols causing cache misses on the first CPU (with \textbf{perf} pinned to the second CPU):
   \begin{raw}
taskset 2 perf top -C 0 -e cache-misses
   \end{raw}

}
